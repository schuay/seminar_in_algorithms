\documentclass[a4paper,10pt]{article}

\usepackage{acronym}
\usepackage[backend=bibtex]{biblatex}
\usepackage{comment}
\usepackage[pdfborder={0 0 0}]{hyperref}
\usepackage[utf8]{inputenc}

\acrodef{CAS}{Compare-And-Swap}
\acrodef{DCAS}{Double-Compare-And-Swap}
\acrodef{DCSS}{Double-Compare-Single-Swap}
\acrodef{DS}{Data Structure}
\acrodef{FAA}{Fetch-And-Add}
\acrodef{FAO}{Fetch-And-Or}
\acrodef{PQ}{Priority Queue}
\acrodef{TAS}{Test-And-Set}

\bibliography{../common/bibliography.bib}

\title{Lock-free skiplist-based priority queues}
\author{Jakob Gruber, 0203440}

\begin{document}

\maketitle

\begin{comment}
Abstract formula
----------------

1) big picture problem or topic widely debated in your field.
2) gap in the literature on this topic.
3) your project filling the gap.
4) the specific material that you examine in the paper.
5) your original argument.
6) a strong concluding sentence.

Introduction formula
--------------------

A good paper introduction is fairly formulaic. If you follow a simple set of
rules, you can write a very good introduction. The following outline can be
varied. For example, you can use two paragraphs instead of one, or you can
place more emphasis on one aspect of the intro than another. But in all cases,
all of the points below need to be covered in an introduction, and in most
papers, you don't need to cover anything more in an introduction.

Paragraph 1: Motivation. At a high level, what is the problem area you are
working in and why is it important? It is important to set the larger context
here. Why is the problem of interest and importance to the larger community?

Paragraph 2: What is the specific problem considered in this paper? This
paragraph narrows down the topic area of the paper. In the first paragraph you
have established general context and importance. Here you establish specific
context and background.

Paragraph 3: "In this paper, we show that ...". This is the key paragraph in
the intro - you summarize, in one paragraph, what are the main contributions of
your paper given the context you have established in paragraphs 1 and 2. What
is the general approach taken? Why are the specific results significant? This
paragraph must be really really good. If you can't "sell" your work at a high
level in a paragraph in the intro, then you are in trouble. As a reader or
reviewer, this is the paragraph that I always look for, and read very
carefully.

You should think about how to structure this one or two paragraph summary of
what your paper is all about. If there are two or three main results, then you
might consider itemizing them with bullets or in test (e.g., "First, ..."). If
the results fall broadly into two categories, you can bring out that
distinction here. For example, "Our results are both theoretical and applied in
nature. (two sentences follow, one each on theory and application)"

Paragraph 4: At a high level what are the differences in what you are doing,
and what others have done? Keep this at a high level, you can refer to a future
section where specific details and differences will be given. But it is
important for the reader to know at a high level, what is new about this work
compared to other work in the area.

Paragraph 5: "The remainder of this paper is structured as follows..." Give the
reader a roadmap for the rest of the paper. Avoid redundant phrasing, "In
Section 2, In section 3, ... In Section 4, ... " etc.

A few general tips:

Don't spend a lot of time into the introduction telling the reader about what
you don't do in the paper. Be clear about what you do do, but don't dwell here
on what you don't do.  Does each paragraph have a theme sentence that sets the
stage for the entire paragraph? Are the sentences and topics in the paragraph
all related to each other?  Do all of your tenses match up in a paragraph?
\end{comment}

\begin{abstract}
blablablu blublubla
\end{abstract}

\section{Introduction}
\section{Related Work}
\section{Conclusion}

% TODO: Remove me.
\nocite{*}

\printbibliography

\end{document}
