\documentclass[a4paper,10pt]{article}

\usepackage{acronym}
\usepackage[backend=bibtex]{biblatex}
\usepackage[usenames,dvipsnames]{color}
\usepackage{comment}
\usepackage[pdfborder={0 0 0}]{hyperref}
\usepackage[utf8]{inputenc}
\usepackage{listings}

\acrodef{CAS}{Compare And Swap}
\bibliography{../common/bibliography.bib}

\definecolor{Gray}{gray}{0.5}
\definecolor{OliveGreen}{cmyk}{0.64,0,0.95,0.40}

\lstset{
    language=C++,
    basicstyle=\ttfamily,
    keywordstyle=\color{OliveGreen},
    commentstyle=\color{Gray},
    captionpos=b,
    breaklines=true,
    breakatwhitespace=false,
    showspaces=false,
    showtabs=false,
    numbers=left,
}

\title{Lock-free Skiplist-based Priority Queues \\
       Student Paper for Seminar in Algorithms 2013W \\
       Technical University of Vienna}
\author{Jakob Gruber, 0203440}

\begin{document}

\maketitle

\begin{comment}
Abstract formula
----------------

1) big picture problem or topic widely debated in your field.
2) gap in the literature on this topic.
3) your project filling the gap.
4) the specific material that you examine in the paper.
5) your original argument.
6) a strong concluding sentence.

Introduction formula
--------------------

A good paper introduction is fairly formulaic. If you follow a simple set of
rules, you can write a very good introduction. The following outline can be
varied. For example, you can use two paragraphs instead of one, or you can
place more emphasis on one aspect of the intro than another. But in all cases,
all of the points below need to be covered in an introduction, and in most
papers, you don't need to cover anything more in an introduction.

Paragraph 1: Motivation. At a high level, what is the problem area you are
working in and why is it important? It is important to set the larger context
here. Why is the problem of interest and importance to the larger community?

Paragraph 2: What is the specific problem considered in this paper? This
paragraph narrows down the topic area of the paper. In the first paragraph you
have established general context and importance. Here you establish specific
context and background.

Paragraph 3: "In this paper, we show that ...". This is the key paragraph in
the intro - you summarize, in one paragraph, what are the main contributions of
your paper given the context you have established in paragraphs 1 and 2. What
is the general approach taken? Why are the specific results significant? This
paragraph must be really really good. If you can't "sell" your work at a high
level in a paragraph in the intro, then you are in trouble. As a reader or
reviewer, this is the paragraph that I always look for, and read very
carefully.

You should think about how to structure this one or two paragraph summary of
what your paper is all about. If there are two or three main results, then you
might consider itemizing them with bullets or in test (e.g., "First, ..."). If
the results fall broadly into two categories, you can bring out that
distinction here. For example, "Our results are both theoretical and applied in
nature. (two sentences follow, one each on theory and application)"

Paragraph 4: At a high level what are the differences in what you are doing,
and what others have done? Keep this at a high level, you can refer to a future
section where specific details and differences will be given. But it is
important for the reader to know at a high level, what is new about this work
compared to other work in the area.

Paragraph 5: "The remainder of this paper is structured as follows..." Give the
reader a roadmap for the rest of the paper. Avoid redundant phrasing, "In
Section 2, In section 3, ... In Section 4, ... " etc.

A few general tips:

Don't spend a lot of time into the introduction telling the reader about what
you don't do in the paper. Be clear about what you do do, but don't dwell here
on what you don't do.  Does each paragraph have a theme sentence that sets the
stage for the entire paragraph? Are the sentences and topics in the paragraph
all related to each other?  Do all of your tenses match up in a paragraph?
\end{comment}

\begin{abstract}
On the 24th of February, 1815, the watch-tower of Notre-Dame de la Garde
signalled the arrival of the three-master Pharaon, from Smyrna, Trieste, 
and Naples.

The usual crowd of curious spectators immediately filled the quay of Fort 
Saint-Jean, for at Marseilles the arrival of a ship is always a great event, 
especially when that ship, as was the case with the Pharaon, has been built, 
rigged, and laden in the dockyard of old Phocaea and belongs to a shipowner 
of their own town.

Meanwhile the vessel drew on, and was approaching the harbour under topsails, 
jib, and foresail, but so slowly and with such an air of melancholy that the 
spectators, always ready to sense misfortune, began to ask one another what 
ill-luck had overtaken those on board. However, those experienced in navigation 
soon saw that if there had been any ill-luck, the ship had not been the 
sufferer, for she advanced in perfect condition and under skilful handling; 
the anchor was ready to be dropped, the bowsprit shrouds loose. 
Beside the pilot, who was steering the Pharaon through the narrow entrance to 
the port, there stood a young man, quick of gesture and keen of eye, who 
watched every movement of the ship while repeating each of the pilot's orders.
\end{abstract}

\section{Introduction}

In the past decade, advancements in computer performance have been made mostly
through increasing the number of processors instead of higher clock speeds.
This development necessitates new approaches to data structures and algorithms
that take advantage of concurrent execution on multiple threads and processors. 

This paper focuses on the priority queue data structure, consisting of two operations
traditionally called \lstinline|Insert| and \lstinline|DeleteMin|. \lstinline|Insert|
places an item together with its priority into the queue, while \lstinline|DeleteMin|
removes and returns the highest priority item. Both operations are expected to have
a complexity of at most $O(\log n)$. Priority queues are used in a large variety
of situations such as shortest path algorithms and scheduling.

Concurrent priority queues have been the subject of research since the 1980s.
While early efforts have focused mostly on parallelizing Heap structures
\cite{hunt1996efficient}, % TODO: Many more, see \cite{shavit2000skiplist}
more recently priority queues based on \citeauthor{pugh1990skip}'s SkipLists
\cite{pugh1990skip} seem to show more potential \cite{shavit2000skiplist,
sundell2003fast,herlihy2012art,linden2013skiplist}.

In the following, we examine the evolution of concurrent priority queues from
an early prominent Heap-based algorithm through initial SkipList-based designs
to current state-

\begin{comment}
Sections / rough structure:
* Basic concepts and definitions. Linearizability, sequ./quiescent consistency,
  lock-free, wait-free, disjoint-access parallelism (one of the papers has good
  summaries of these).
  Maybe atomic primitives such as CAS (but probably not).
* Priority queue definitions, semantics, usages. Mention inherent
  non-scalability through DeleteMin().
* Skiplist, heap definitions, semantics, usages.
* Follow development of current state of the art from Hunt Heap ~> Shavit ~>
  Tsigas ~> Linden.
* Maybe benchmarks on mars.
\end{comment}

\section{Related Work}

\begin{comment}
  * hunt: most efficient old-school algorithm
  * lots of other older references in [4]
  * israeli, rappoport: wait-free, non-available atomic primitive
  * lotan, shavit [4]: skiplist, lock-based. first to propose usage of skip-lists [11]
  * sundell, tsigas [3]: skiplist, lock-free, linearizable, unique priorities
  * herlihy, shavit in art of multiprocessor programming: based on [4], lock-free.
  * linden, jonsson [11]: skiplist, lock-free, linearizable, reduced mem contention.
    first lock-free PQ algorithm [11, 12]
  * specialized versions such as
    * bounded priorities: [7] and others
    * probabilistic extractMin, extractMany: [9]
\end{comment}

\section{Conclusion}

\begin{comment}
http://leo.stcloudstate.edu/acadwrite/conclude.html
http://writingcenter.unc.edu/handouts/conclusions/
http://www.wikihow.com/Write-a-Conclusion-for-a-Research-Paper
\end{comment}

% TODO: Remove me.
\nocite{*}

\printbibliography

\end{document}
