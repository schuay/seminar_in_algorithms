\documentclass[usenames,dvipsnames]{beamer}

\usepackage{acronym}
\usepackage[backend=bibtex]{biblatex}
\usepackage{default}
\usepackage[utf8]{inputenc}
\usepackage{listings}
\usepackage{lmodern}
\usepackage{textcomp}

\acrodef{CAS}{Compare-And-Swap}
\acrodef{DCAS}{Double-Compare-And-Swap}
\acrodef{DCSS}{Double-Compare-Single-Swap}
\acrodef{DS}{Data Structure}
\acrodef{FAA}{Fetch-And-Add}
\acrodef{FAO}{Fetch-And-Or}
\acrodef{PQ}{Priority Queue}
\acrodef{TAS}{Test-And-Set}

\bibliography{../common/bibliography.bib}

\setbeamertemplate{bibliography item}{}

\lstset{
    language=C++,
    basicstyle=\ttfamily,
    keywordstyle=\color{OliveGreen},
    commentstyle=\color{Gray},
    captionpos=b,
    breaklines=true,
    breakatwhitespace=false,
    showspaces=false,
    showtabs=false,
    numbers=none,
}

\title{Concurrent Priority Queues}
\subtitle{Seminar in Algorithms, 2013W}
\author{Jakob Gruber, 0203440}
\date{\today}

\begin{document}

\maketitle

\begin{frame}{Outline}
\begin{minipage}[t][10em][t]{\linewidth}
\tableofcontents
\end{minipage}
\end{frame}

% --------------------------------------------------------------------------------------------------
\section{Introduction}
% --------------------------------------------------------------------------------------------------

\begin{frame}[fragile,allowframebreaks]{Introduction}
\acp{PQ}:

\begin{itemize}
\item Standard abstract data structure
\item Used widely in algorithmics, operating systems, task scheduling, etc
\item Interface consists of two $O(\log n)$ operations:

\begin{lstlisting}
void Insert(pq_t *pq, key_t k, value_t v)
bool DeleteMin(pq_t *pq, value_t *v)
\end{lstlisting}

\item Typical backing data structures: heaps \& search trees
\end{itemize}

\framebreak

\begin{itemize}
\item In the past decade, processor clock speeds have remained the same, trend towards multiple cores
\item New data structures required to take advantage of concurrent execution
\item The topic of this presentation: efficient concurrent \acp{PQ}
\item Fine-grained locking \textrightarrow ~ Lock-free \textrightarrow ~ Relaxed data structures
\end{itemize}

\end{frame}

\begin{frame}[allowframebreaks]{Terminology}
\begin{itemize}
\item Safety conditions: nothing bad has happened yet
    \begin{itemize}
    \item \emph{Linearizability}: operations appear to take effect at a single point in time, the linearization point
    \item \emph{Quiescent consistency}: in a period of quiescence, semantics equivalent to some sequential ordering
    \item And others, i.e. \emph{sequential consistency}, \emph{serializability}
    \end{itemize}

\framebreak

\item Liveness conditions: something good eventually happens
    \begin{itemize}
    \item \emph{Lock-freedom}: at least a single process makes progress at all times
    \item \emph{Wait-freedom}: every process finishes in a bounded number of steps
    \item Further liveness conditions: \emph{Starvation-freedom}, \emph{Deadlock-freedom}
    \end{itemize}

\framebreak

\item \emph{Disjoint-access parallelism}: how well a data structure handles concurrent use by multiple
      threads within disjoint areas
\item Synchronization primitives:
    \begin{itemize}
    \item \ac{CAS}, \ac{FAA}, \ac{FAO}, \ac{TAS}
    \item \ac{DCAS}, \ac{DCSS}
    \end{itemize}

\end{itemize}

\end{frame}

% --------------------------------------------------------------------------------------------------
\section{Related Work}
% --------------------------------------------------------------------------------------------------

\begin{frame}{Related Work}
\begin{itemize}
\item Non-standard synchronization primitives
    \begin{itemize}
    \item \citeauthor{liu2012lock}: Array-based PQ with \lstinline|ExtractMany|
    \item \citeauthor{israeli1993efficient}: Wait-free PQ
    \end{itemize}

\item Bounded range priorities
    \begin{itemize}
    \item \citeauthor{shavit1999scalable}: Combining funnels \& bins
    \end{itemize}

\item PQs in distributed memory systems
    \begin{itemize}
    \item \citeauthor{sanders1998randomized}: Local PQs, \lstinline|Insert| at random processor
    \end{itemize}

\item Relaxed data structures
    \begin{itemize}
    \item \citeauthor{kirsch2012fast}: k-FIFO queues
    \end{itemize}

\end{itemize}
\end{frame}

% --------------------------------------------------------------------------------------------------
\section{Concepts and Definitions}
% --------------------------------------------------------------------------------------------------

% --------------------------------------------------------------------------------------------------
\section{Fine-grained Locking Heaps}
% --------------------------------------------------------------------------------------------------

% --------------------------------------------------------------------------------------------------
\section{Lock-free Priority Queues}
% --------------------------------------------------------------------------------------------------

% --------------------------------------------------------------------------------------------------
\section{Relaxed Priority Queues}
% --------------------------------------------------------------------------------------------------

% --------------------------------------------------------------------------------------------------
\section{Conclusion}
% --------------------------------------------------------------------------------------------------

% --------------------------------------------------------------------------------------------------
\section{References}
% --------------------------------------------------------------------------------------------------

\begin{frame}[allowframebreaks]{References}
\printbibliography
\end{frame}

\end{document}
